\resetfigpath{Chap6}

%%%{INTRODUCTION to this CHAPTER}%%

%%%%%%%%%%%%%%%%%%%%%%%%%%%%%%%%%%%%%%%%%%%%%%%%%%%%%%%%%%%%%%%%%%%%%%%%
%%%%%%%%%%               SECTION SECTION SECTION               %%%%%%%%%
%%%%%%%%%%%%%%%%%%%%%%%%%%%%%%%%%%%%%%%%%%%%%%%%%%%%%%%%%%%%%%%%%%%%%%%%

In this research, the behaviors of human drivers were modeled in a probability measure of yield based on the cognitive information. The proposed POY enables a formal formulation of human driving decision processes. Experiments conducted in both simulated and urban environments also support the idea by showing predictions similar to that of human drivers. The classification accuracy rate using the proposed model is also comparable to the state-of-the-art method.

We extend the proposed model to traffic risk assessments and driver behavior parameters identifications. Under reasonable assumptions the preliminary attempts to assess the safety of crossroads as well as to identify different driving style are explored. These possible applications show how the proposed model could be further developed to help improving the road safety using the TFA distribution and the prediction results.

The contexts of the presented work assume 1) the TTC is the major role in human drivers' cognitive behaviors in the simplified crossroad model; 2) the TFA being normally distributed under available verification. In the future, the proposed model will be extended for multi-agent joint behaviors predictions to form a more realistic prediction model. The parameter identification would also be modified to enhance the reliability and robustness of parameter identification. The average parameters of an area can not only gives us the driving styles of the area, but also help autonomous vehicles with adjusted judgements of the possible behaviors of the neighborhood. The work presented can help building a computer driving logic that matches human behaviors such that interactions between different drivers will be more intuitive.