無人車與人類駕駛的互動與溝通,將會在不遠的將來成為主要的議題。在這篇論文中,本論文將研究聚焦於最仰賴互動的無交通號誌十字路口。為了研究駕駛是如何進行決策來安全通過路口,本論文首先定義了路口的重要參數,包括了互動車輛的速度與離路口的距離,並定義了一個駕駛決策行為。藉由研究此決策行為,本論文得到了互動車輛煞車讓道的機率,以及駕駛行為的機率模型。為了驗證此模型,本論文在模擬環境中進行測試,並蒐集真人駕駛的數據進行分析。所得到的驗證結果證明,所提出模型的預測準確率與現有方法接近,且具有更廣泛的應用,同時此預測模型也能夠反映出不同駕駛特徵參數的差異。接著,本論文嘗試使用最佳化方法,藉由所蒐集數據進行駕駛行為的特徵參數回推。儘管此參數辨認的準確率尚有改進空間,目前所得到結果證明了所提出模型在此應用上的可行性。同時,為了驗證此模型在實際路口的可行性,本論文蒐集了真實路口的車輛數據並使用所提出模型進行行為預測,所得結果與模擬環境中的結果一致。最後,本論文將所提出模型應用在無人車的決策行為上並與人類駕駛進行互動,初步結果證明無人車的駕駛行為能夠更順暢的與人類駕駛進行互動。


~\\

\textbf{關鍵字}: 混流, 無人駕駛, 機率模型, 駕駛行為, 十字路口, 互動模型, 避障.