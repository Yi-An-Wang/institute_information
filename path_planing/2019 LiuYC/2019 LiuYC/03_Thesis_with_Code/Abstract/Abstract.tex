The interactions with human drivers is one of the major challenges for autonomous vehicles in the near future. In this work we consider urban crossroads without signals where driver interactions are indispensable. Crossroad parameters are defined and how drivers passing the crossroad while maintaining a desired speed without collision is studied. A point of action is defined for incoming vehicles from different directions and a probability of yielding for each car is proposed as a function of vehicle speed and the distance-to-intersection for both vehicles. Driver behaviors with these probability models are also proposed. The method is then analyzed and validated by data collected from human drivers in the simulated environments. The result shows comparable prediction accuracy to the state of the art method, where characteristic parameters of drivers are also shown to be critical for the behavior predictions. Afterwards, parameters representing driving styles of drivers are attempted to identify using the optimization approach. In spite of the limited accuracy of parameter identifications, important attributes of the proposed model as well as possible modification are pinpointed. The proposed model is also applied at the urban crossroads to evaluate the applicability in real world. The prediction results are analogous to those acquired in virtual environments. Finally, a procedure is constructed to achieve smoother interactions with human drivers. Preliminary results suggested a human-like computer driver is born while more instances and aspects of evaluations should be accomplished in the future work.

~\\

\textbf{Keywords}: mixed-fleet, autonomous vehicle, probabilistic model, driver behavior, crossroad, interaction model, collision avoidance.